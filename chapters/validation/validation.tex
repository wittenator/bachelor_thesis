% ---------------------------------------------------
% ----- Conclusion of the template
% ----- for Bachelor-, Master thesis and class papers
% ---------------------------------------------------
%  Created by C. Müller-Birn on 2012-08-17, CC-BY-SA 3.0.
%  Freie Universität Berlin, Institute of Computer Science, Human Centered Computing. 
%
\chapter{Validation}
\label{chap:validation}

In order to show how the implemented explainability techniques may help a non-technical expert understanding the output generated by the pipeline, a proper user study would be needed. Since that is a task which would fill a bachelor thesis on its own, I resorted to a strategy which involves less work, but also delivers qualitative insights into the interaction between a user and the topic modeling pipeline and its visualization - the cognitive walkthrough. Performing this method involves seeing things from the perspective of a fictive user and interacting with the application in their stead.
Since the nature of this visualization supports exploratory interactions in the first place, standard approaches for cognitive walkthroughs, like they are formulated in \cite{whartonUsabilityInspectionMethods1994}, do not work well due to the necessity of coding an interaction sequence prior to the simulated interaction. Allendorf et al. \cite{allendoerferAdaptingCognitiveWalkthrough2005} adapted the well-established method of cognitive walkthroughs to this kind of use case. Their method consists of defining a persona, goals for the interaction and possible steps which can be taken in the visualization. With this setup an action is performed which seems most applicable to reach the current goal and afterwards the following four questions are answered:
\begin{enumerate}
	\item What effect was the user trying to achieve by selecting this action?
	\item How did the user know that this action was available?
	\item Did the selected action achieve the desired effect?
	\item When the action was selected, could the user determine how things were going?
\end{enumerate}

\section{Setup}
The fictive user is a postdoctoral researcher at the Museum für Naturkunde Berlin. His background is characterized by the following features:
\begin{itemize}
	\item \textbf{Education}
	Is a postdoctoral researcher of biology specializing in evolutionary theory
	
	\item \textbf{Relevant work experience}
	Currently working on a project called "Variabilität von MHC-Genen bei der Sackflügelfledermaus Saccopterix bilineata" investing the genetic variability of bats
	
	\item \textbf{Experience with user interface design and usability assessment}
	Has no prior knowledge of interface design or usability assessment
	
	\item \textbf{Operating systems and software packages used frequently}
	Microsoft Windows; Microsoft Office (Word, Excel, PowerPoint); Microsoft Outlook; Mozilla Firefox; Microsoft Media Player; LaTeX; Zotero
\end{itemize}

As described in the Introduction, there are no common meeting rooms for the scientific staff at the museum. The interface is therefore positioned on a location which has the biggest throughput in the museum - in this case the side entrance which is exclusively used by the museum's staff.
The scenario and the context for this exemplary interaction is as follows:

	 One day after work, the fictive user is coming down the wide stairway of the side building he works in and sees once more the display with the visualization he passes every day on his way to and from work. This time the curiosity is stronger than the urge to go home and since he already heard that the museum financed a huge initiative to foster intra-organizational, scientific exchange, he decides to see what that application has to offer.


Looking at the questions formulated in the beginning of this thesis, we can now derive specific tasks this user may want to complete to answer the questions:
\begin{enumerate}
	\item Identify dominating research areas
	\item Find his own project
	\item Explore the projects in the same cluster
	\item Explore the projects in the vicinity of his own cluster
\end{enumerate}

The prototypical interface provides the following actions in order to manipulate the visualization:
\begin{enumerate}
	\item Investigate metadata for a project (title, ID and top words)
	\item Investigate top words for all clusters
	\item Change the number of clusters
	\item Switch between the linearized view and the scatter view
\end{enumerate}



\section{Cognitive Walkthrough} 

This cognitive walkthrough shows that the implemented explainability techniques help a non-technical expert to navigate in the visualization as well as understand why certain projects are located as they are since almost every step utilizes at least one of the techniques. Furthermore it unveiled a number of usability issues with the visualization. The first found problem is dismissable since the presented interface is not the final one which is going to go live, but the second may hinder the interaction with the visualization even in the actual prototype.

\newpage

\begin{longtable}[l]{| p{.4\textwidth} | p{.6\textwidth}  |} 
	\hline
	Description & Usability Impact \\ \hline \hline
	The label for the dropdown selection between the scatter plot and the linearized view does not properly describes what it does. & 
	Uncertainty about the usage of a tool may disturb a user in the inference task, therefore a descriptive name for this selection should be chosen. \\ \hline
	There is no visual connection between views while changing the cluster or view parameters. &
	Perturbing these parameters does not change the underlying displayed corpus, but the recomputation of the full pipeline may lead, due to the random initialization of the K-Means algorithm, to dramatically different outputs. Tracking these changes is quite hard and therefore after each change the user has to orient himself in the visualization anew. Adding animated transitions could help alleviating this problems by introducing object permanence in the views. \\ \hline
	
	
	\caption{Table summarizing the found usability design issues} 
	\label{tab:usability_problems}
\end{longtable}
