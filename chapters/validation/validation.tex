% ---------------------------------------------------
% ----- Conclusion of the template
% ----- for Bachelor-, Master thesis and class papers
% ---------------------------------------------------
%  Created by C. Müller-Birn on 2012-08-17, CC-BY-SA 3.0.
%  Freie Universität Berlin, Institute of Computer Science, Human Centered Computing. 
%
\chapter{Validation}
\label{chap:validation}


Since the nature of this visualization supports exploratory interactions in the first place, standard approaches for cognitive walkthroughs, like they are formulated in \cite{whartonUsabilityInspectionMethods1994}, do not work well due to the necessity of coding an interaction sequence prior to the simulated interaction. Allendorf et al. \cite{allendoerferAdaptingCognitiveWalkthrough2005} adapted the well-established method of cognitive walkthroughs to this kind of use case. Their method consists of defining a persona, goals for the interaction and possible steps which can be taken in the visualization. With this setup an action is performed which seems most applicable to reach the current goal and afterwards the following four questions are answered:
\begin{enumerate}
	\item What effect was the user trying to achieve by selecting this action?
	\item How did the user know that this action was available?
	\item Did the selected action achieve the desired effect?
	\item When the action was selected, could the user determine how things were going?
\end{enumerate}

\section{Setup}
The fictive user is a postdoctoral researcher at the Museum für Naturkunde Berlin. His background is characterized by the following features:
\begin{itemize}
	\item \textbf{Education}
	Is a postdoctoral researcher of biology specializing in evolutionary theory
	
	\item \textbf{Relevant work experience}
	Currently working on a project called "Variabilität von MHC-Genen bei der Sackflügelfledermaus Saccopterix bilineata" investing the genetic variability of bats
	
	\item \textbf{Experience with user interface design and usability assessment}
	Has no prior knowledge of interface design or usability assessment
	
	\item \textbf{Operating systems and software packages used frequently}
	Microsoft Windows; Apple OS X; Microsoft Office
	(Word, Excel, PowerPoint); Microsoft Outlook; Mozilla Firefox;
	Microsoft Media Player; LaTeX; Zotero
\end{itemize}

As described in the Introduction, there are no common meeting rooms for the scientific staff at the museum. The interface is therefore positioned on a location which has the biggest throughput in the museum - in this case the side entrance which is exclusively used by the museum's staff. One day after work, the fictive user is coming down the wide stairway of the side building he works in and sees once more the display with the visualization he passes every day on his way to and from work. This time the curiosity is stronger than the urge to go home and since he already heard that the museum financed a huge initiative to foster intra-organizational, scientific exchange, he decides to see what that application has to offer.

Looking at the questions formulated in the beginning of this thesis, we can now derive specific tasks this user may want to complete to answer the questions:
\begin{enumerate}
	\item Identify dominating research areas
	\item Find his own project
	\item Explore the projects in the same cluster
	\item Explore the projects in the vicinity of his own cluster
\end{enumerate}

The prototypical interface provides the following actions in order to manipulate the visualization:
\begin{enumerate}
	\item Investigate metadata for a project (title, ID and top words)
	\item Investigate top words for all cluster
	\item Change the number of clusters
	\item Switch between the linearized view and the scatter view
\end{enumerate}



\section{Cognitive Walkthrough} 

{\fontsize{9}{11}\selectfont
\begin{longtable}[l]{| p{.1\textwidth} | p{.2\textwidth} | p{.2\textwidth} | p{.2\textwidth} | p{.2\textwidth} |} 
    \hline 
	Action & 1 & 2 & 3 & 4 \\ \hline \hline
	2 (\autoref{pic:step1}) & 
	The user sees the visualization for the first time and tries to connect the cluster top words, which are displayed above the visualization and the clusters in the visualization. &
	This action follows from the immediate presentation of top words and scatter plot. &
	The user deduces that all the projects can be clustered into three clusters - a taxonomic cluster, one connected to ecosystems and one concerned with evolution. &
	This action did not change the visualization. \\ \hline
	3 (\autoref{pic:step2}) &
	The user concluded that his project must be in the 'evolution' cluster and therefore he changes the granularity by moving the slider to the middle of the selection range. &
	It was the only available slider and its label suggested that this is the proper action. &
	The user sees a more granular clustering over all projects. &
	Since there is no transition, the user does not know what is happening. \\ \hline
	2 (\autoref{pic:step2}) &
	After changing the granularity, the user is trying to pinpoint the cluster to which his project is now assigned. &
	As in the beginning the immediate presentation of the topwords and the visualization makes it inevitable to read and connect both. &
	The user determined that his project is probably in cluster 4. &
	This action did not change the visualization. \\ \hline
	1 (\autoref{pic:step3}) & 
	The user is trying to find his project in cluster 4. &
	Hovering over a glyph to display metadata is a common design strategy and therefore the user tried this first. &
	The selected project was not the correct one. &
	Since the metadata is displayed right alongside the project and the rest didn't change, there was no confusion. \\ \hline
	1 (\autoref{pic:step4}) & 
	The user is trying to find his project in cluster 4. &
	Hovering over a glyph to display metadata is a common design strategy and therefore the user tried this first. &
	The selected project was again not the correct one. &
	Since the metadata is displayed right alongside the project and the rest didn't change, there was no confusion. \\ \hline
	4 (\autoref{pic:step5}) & 
	Since the rest of the cluster is extremely cluttered, the user decides to switch into the linearized view &
	This is the only remaining interaction option, but the naming of this selection makes it hard to intuitivly understand what it does. &
	The scatter plot got uncluttered by linearization. &
	Since there is no animated transition, the user does not really know how the scatter plot and this view are connected. Furthermore the cluster assignments and the top words changed, because the whole pipeline was computed from ground up. This adds into the confusion. \\ \hline
	(\autoref{pic:step5}) &
	Now that the clustering and assignments changed again, the user is again searching for the cluster in which his project may lie. &
	As in the previous instances of this action this is the only available, logical action. &
	The user is able to pinpoint cluster 3 as the cluster connected to bats. &
	Nothing changed, therefore there was no room for confusion. \\ \hline
	1 (\autoref{pic:step6}) &
	The user is searching for his project and selects the first visible glyph. &
	In the previous steps the user verified that hovering for metadata is a possibility. &
	The project he selected was indeed his own project. All of the top words make sense in the context of his project. &
	Again there was no confusion. \\ \hline
	1 (\autoref{pic:step7}) &
	Now that he found his project the user is interested what kind of projects are also in the cluster which was assigned to his project. Therefore he selects the next available project in the same cluster. &
	In the previous steps the user verified that hovering for metadata is a possibility. &
	The next project is also connected to the very same research subject. Therefore the clustering makes sense and the top words also are closely related to the top words of his own project. &
	Again there was no confusion. \\ \hline
	1 (\autoref{pic:step8}) &
	The user selects the last remaining project in the same cluster to see if it also fits into the cluster since the topography suggests that it may fit less well into the overarching topic. &
	In the previous steps the user verified that hovering for metadata is a possibility. &
	The last project is also connected to a similar research subject, although it differs a bit due to it rather being concerned with migration of bats than procreation. Since the topwords also suggest this, it further validates the clustering. &
	Again there was no confusion. \\ \hline
	1 (\autoref{pic:step9}) &
	Since the cluster does make sense the user decides to have a look at the neighbouring projects. &
	In the previous steps the user verified that hovering for metadata is a possibility. &
	The first project he selects does investigate a completly different field. &
	The user is able to tell why the project lies in another cluster using the top words. \\ \hline
	1 (\autoref{pic:step10}) &
	The user still thinks that there may be another similar project, because the top words of cluster 2 are a concerned with ecology, faunas and taxonomies. &
	In the previous steps the user verified that hovering for metadata is a possibility. &
	The next project he selects is surprisingly also connected to bats.  &
	The user is not entirely sure why this project was not categorized in his own cluster. The top words suggest that the work is rather specialzed on the biological processes of procreation using bats as a use case.  \\ \hline
	
	
	
	
	
	\caption{Exploratory interaction simulated by a CW} 
	\label{tab:steps_cw}
\end{longtable}}

This cognitive walkthrough unveiled a number of usability issues with the visualization, but also showed that the implemented explainability techniques do indeed help the fictive used to accomplish his goals. 

\newpage

\begin{longtable}[l]{| p{.4\textwidth} | p{.6\textwidth}  |} 
	\hline
	Description & Usability Impact \\ \hline \hline
	The label for the dropdown selection between the scatter plot and the linearized view does not properly describes what it does. & 
	Uncertainty about the usage of a tool may disturb a user in the inference task, therefore a descriptive name for this selection should be chosen. \\ \hline
	There is no visual connection between views while changing the cluster or view parameters. &
	Perturbing these parameters does not change the underlying displayed corpus, but the recomputation of the full pipeline may lead, due to the random initialization of the K-Means algorithm, to dramatically different outputs. Tracking these changes is quite hard and therefore after each change the user has to orient himself in th visualization anew. Adding animated transitions could help alleviating this problems by introducing object permanence in the views. \\ \hline
	
	
	\caption{Table summarizing the found usability design issues} 
	\label{tab:usability_problems}
\end{longtable}
